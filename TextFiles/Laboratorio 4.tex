\chapter{Laboratorio 4}

Il primo circuito analizzato è il multivibratore monostabile (\Fig\ref{fig:circuito_1}). Tale circuito presenta uno stato stabile, in cui rimane fino a che un impulso esterno di comando lo porta in un secondo stato, in cui il circuito rimane per un tempo predefinito per poi tornare spontaneamente nello stato stabile, in attesa di un successivo impulso.
\begin{figure}[h]
	\centering
	\begin{minipage}{.45\textwidth}
		\scalebox{.62}{
			\begin{circuitikz}
				\draw (2,6) node[op amp, anchor=-](oa){\texttt{TL071}};
				\draw (oa.-) -- ++(-2, 0) coordinate (D) -- ++(-2,0) to[C=$C$] ++(0,-1.5) node[ground]{};
				\draw (D) to[D=$D$] ++(0,-1.5) node[ground]{};
				\draw (oa.up) -- ++(0, 0.3) node[vcc]{$V_{DD}$};
				\draw (oa.down) -- ++(0,-0.3) node[vee]{$V_{SS}$};
				\draw (oa.out) -- ++(1,0) coordinate(loop);
				\draw (loop) -- ++(0,-2.3) coordinate(R2) to[R=$R_2$] ++(-3.39,0) coordinate(rg) (R2-|oa.+) -- (oa.+);
				\draw (oa.-) -- ++(0,1.7) to[R=$R_3$]++(3.39,0) coordinate(Rf) (Rf-|loop) -- (loop);
				\draw (oa.+) to[short, -o] ++(0,0) ++(-.1,0) node[left]{$v_+$};
				\draw (oa.-) to[short, -o] ++(0,0) ++(0,-.1) node[below]{$v_-$};
				\draw (rg) -- ++(-1,0) coordinate(r1) to[R=$R_1$] ++(0,-2) node[ground]{};
				\draw (r1) to[D=$D_T$] ++(-2.5,0) coordinate(vt) to[C=$C_T$] ++(-3,0) coordinate(vin);
				\draw (vt) to[R=$R_T$] ++(0,-2) node[ground]{};
				\draw (vt) to[short, -o] ++(0,0) node[above]{$v_t$};
				\draw (loop) to[short, -o] ++(0,0) ++(.1,0) node[right]{$v_{out}$};
				\draw (vin) to[sV=$v_{in}$] ++(0,-2) node[ground]{};
				\draw[thick] (-5.5,0) rectangle (6.5,8.5);
			\end{circuitikz} 
		}
	\end{minipage}\qquad
	\begin{minipage}{.45\textwidth}
		\includegraphics[width=\linewidth]{./ImageFiles/Laboratorio 4/CIR11.jpg}
	\end{minipage}
	\caption{Schema circuitale del multivibratore monostabile e foto del circuito realizzato.}
	\label{fig:circuito_1}
\end{figure}
A livello costruttivo si parte dal generatore d'onda quadra visto nello scorso laboratorio: inserendo un diodo $D$ in parallelo alla capacità $C$ si ottiene uno stato stabile. In questo casoSI, quando la tensione di uscita è pari a $v_{out}>0$ la capacità si carica, ma quando la tensione ai capi del condensatore raggiunge circa $0.7V$ (tensione di polarizzazione del diodo $D$) smette di caricarsi perché il diodo offre un cortocircuito per la corrente fornita dall'amplificatore. 

\noindent
Poiché il circuito commuti dallo stato stabile $v_{out}=V_{DD}$ allo stato quasi stabile $v_{out}=V_{SS}$ è necessario che la tensione dell'ingresso non invertente scenda al di sotto della tensione applicata all'ingresso invertente. Questo è possibile tramite l'applicazione sul morsetto non invertente di un picco di tensione negativo. Per fare ciò si inserisce un circuito derivatore composto da una capacità $C_T$ ed una resistenza $R_T$. Tale circuito ha il compito di generare, a partire da un segnale rettangolare $v_in$, un impulso negativo ed uno positivo. Poi, grazie all'inserimento del diodo $D_T$, l'unico impulso presentato al morsetto non invertente è quello negativo. 

\noindent
Questo impulso negativo porterà per un breve istante la tensione $v_+$ al di sotto della tensione presente ai capi del condensatore (che è pari alla tensione di polarizzazione del diodo) forzando la commutazione dell'uscita a $v_{out}=V_{SS}$. A questo punto, il condensatore si scarica fino a raggiungere la tensione di soglia $V_H^-$ e una volta raggiunta la tensione di soglia cambia nuovamente e l'uscita commuta portandosi a $v_{out}=V_{DD}$. Il condensatore ricomincia così il ciclo di carica ma una volta raggiunta la tensione di polarizzazione del diodo si blocca e il circuito entra di nuovo in uno stato stabile in attesa che sul morsetto non invertente venga generato un impulso negativo. Il tempo per cui $v_{out}=V_{SS}$ dipende dalla capacità $C$ e dalla resistenza $R3$.

I valori utilizzati nel nostro circuito sono indicati nella tabella \ref{tab:valori_componenti_1}. Inoltre, gli amplificatori sono stati alimentati con una tensione duale di $\pm$\SI{10}{\volt}.

\def\arraystretch{1.3}
\begin{table}[h]
	\centering
	\begin{tabular}{|c|c|c|}
		\hline
		Componente	& Valore Nominale & Valore Misurato \\ \hline
		R1 &\SI{18}{\kilo\ohm} & \SI{17,96}{\kilo\ohm} \\ \hline
		R2 &\SI{18}{\kilo\ohm} & \SI{17,79}{\kilo\ohm} \\ \hline
		R3 & \SI{18}{\kilo\ohm} & \SI{18}{\kilo\ohm} \\ \hline
		R\sub{T} & \SI{18}{\kilo\ohm} & \SI{17,83}{\kilo\ohm} \\ \hline
		D\sub{T} & $\simeq$\SI{0.7}{\volt} & \SI{609}{\milli\volt} \\ \hline
		D1 & $\simeq$\SI{0.7}{\volt} & \SI{607}{\milli\volt} \\ \hline
		C & \SI{150}{\nano\farad} & Non misurato \\ \hline
	\end{tabular}
	\caption{Valori nominali e misurati dei componenti utilizzati nel circuito.}
	\label{tab:valori_componenti_1}
\end{table}

Il segnale in ingresso utilizzato è un'onda quadra con \textit{duty-cicle} pari al 20\%, frequenza di \SI{100}{\hertz} e ampiezza picco-picco pari a \SI{5}{\volt}\todo{Inserire valori corretti}. In figura \ref{fig:picchi_ingresso} si vedono i picchi generati dal derivatore. 

\begin{figure}[h!]
	\centering
	\includegraphics[width=\linewidth]{./ImageFiles/Laboratorio 4/TEK00002.PNG}
	\caption{Misure del segnale $v_{+}$ (linea gialla). Questi picchi sono generati dal circuito derivatore con in ingresso un'onda quadra con \textit{duty-cicle} del 20\%, frequenza di \SI{100}{\hertz} e ampiezza picco-picco pari a \SI{5}{\volt}.}
	\label{fig:picchi_ingresso}
\end{figure}

In figura \ref{fig:segnale_uscita} sono invece mostrati gli andamenti di $v_{-}$ e $v_{out}$.
\begin{figure}[h!]
	\centering
	\includegraphics[width=\linewidth]{./ImageFiles/Laboratorio 4/TEK00004.PNG}
	\caption{Misure del segnale $v_{-}$ (linea gialla) e del segnale $v_{out}$ (linea azzurra).}
	\label{fig:segnale_uscita}
\end{figure}