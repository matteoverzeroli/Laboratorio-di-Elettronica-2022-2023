\chapter{Laboratorio 1}
In questa esperienza di laboratorio si è realizzato e analizzato il seguente circuito:
\begin{figure}[h!]
	\centering
	\begin{minipage}{.4\textwidth}
	\scalebox{.835}{
		\begin{circuitikz}
			\draw (0,.5) node[ground]{};
			\draw (0,2) to[sV=$v_{in}$] (0,.5);
			\draw (0,2) to[R=$R_2$, -*] ++(3,0) ++(0.1,-.1) node[below]{$V^-$};
			\draw (3,2) to (4,2);
			\draw (3.7,2) node[op amp, anchor=-](oa){\texttt{TL071}};
			\draw (oa.up) -- ++(0, 0.3) node[vcc]{$+E$};
			9 \draw (oa.down) -- ++(0,-0.3) node[vee]{$-E$};
			\draw (3.7,.5) node[ground]{} to[short, -*] (3.7,1);
			\draw (3,2) -- ++(0,2.3) coordinate(C) to[C=$C_1$] ++(3.08,0) (C-|oa.out) coordinate(Co) -- (oa.out) to [short, *-o] ++(1,0) node[above]{$v_{out}$};
			\draw (3,4.3) -- ++(0,1.3) to[R=$R_1$] ++(3,0) -| (Co);
			\draw[thick] (-.7,-.5) rectangle (7.65,6.5);
		\end{circuitikz}
	}
	\end{minipage}
	\qquad\qquad
	\begin{minipage}{.5\textwidth}
		\includegraphics[width=\linewidth]{./ImageFiles/Laboratorio 1/CIRC.jpg}
	\end{minipage}
	\caption{Schema circuitale e foto circuito}
	\label{fig:circuito}
\end{figure}

Il circuito realizza un filtro passa basso attivo del primo ordine, utilizzando un amplificatore operazionale retroazionato negativamente. Infatti, è possibile ricavare la funzione di trasferimento del circuito tramite un bilancio delle correnti al nodo V\super{-}. Si consideri v\sub{in} come un generatore ideale di tensione applicato in ingresso al circuito. Indicando con Z\sub{1} l'impedenza equivalente del parallelo tra R\sub{1} e C\sub{1} e con Z\sub{2} l'impedenza della resistenza R\sub{2}, si può ottenere la funzione di trasferimento del circuito:
\begin{equation}
	v_{out}=-\frac{Z_1}{Z_2}v_{in}=-\frac{R_1}{R_2}\frac{1}{1+j w R_1 C_1} v_{in},
\end{equation}
da cui si ricava
\begin{equation}
	T=\frac{v_{out}}{v_{in}}=-\frac{R_1}{R_2}\frac{1}{1+j w R_1 C_1}.
\end{equation}
Questa funzione di trasferimento corrisponde a un filtro passa basso. È possibile calcolare il valore del modulo e della fase in funzione di $\omega$ tramite le seguenti espressioni:
\begin{equation}
	\begin{split}
		|T|&=\frac{R_1}{R_2}\frac{1}{\sqrt{1+(wR_1C_1)^2}} \\
		\angle T&=\SI{180}{\degree}-arctan(\omega R_1 C_1).
	\end{split}
	\label{eq:1.3}
\end{equation}
Per comprendere l'andamento del modulo e della fase in funzione della frequenza del segnale applicato in ingresso, è necessario analizzare i termini dipendenti da $\omega$ nelle due equazioni. Nell'espressione del modulo della funzione di trasferimento, il termine 
\begin{equation}
	\sqrt{1+(wR_1C_1)^2} \to
	\begin{cases}
		1 \; per \; \omega \to 0 \\
		\infty \; per \; \omega \to \infty
	\end{cases}
,
\end{equation}
mentre nell'espressione della fase, il termine
\begin{equation}
	arctan(\omega R_1 C_1) \to
	\begin{cases}
		0 \; per \; \omega \to 0 \\
		90^\circ \; per \; \omega \to \infty
	\end{cases}
	.
\end{equation}
Il modulo quindi tende a $\frac{R_1}{R_2}$ per segnali in ingresso a bassa frequenza, mentre tende a zero per segnali in ingresso ad alta frequenza, mentre la fase è pari a circa \SI{180}{\degree} per segnali a bassa frequenza, e tende a \SI{90}{\degree} per segnali ad alta frequenza.
Il comportamento del circuito è quindi quello di un filtro passa basso invertente del primo ordine, con frequenza di taglio pari a $f=\frac{1}{2\pi R_1C_1}$.

\noindent
I valori dei componenti passivi sono stati scelti per soddisfare i requisiti di prestazioni del filtro. In particolare, veniva richiesto un guadagno di un fattore 10 con frequenza di taglio pari a \SI{10}{\kilo\hertz}. I valori nominali ed effettivi dei componenti passivi sono riportati in tabella \ref{tab:valori_componenti}. Inoltre, le tensioni di alimentazione positiva +E e negativa -E dell'amplificatore operazionale sono state fissate rispettivamente a +\SI{10}{\volt} e -\SI{10}{\volt}.

\def\arraystretch{1.3}
\begin{table}[h]
	\centering
	\begin{tabular}{|c|c|c|}
		\hline
		Componente	& Valore Nominale & Valore Misurato \\ \hline
		R\sub{1}          & \SI{38}{\kilo\ohm} &     \SI{38.10}{\kilo\ohm}  \\ \hline
		R\sub{2}          & \SI{3.3}{\kilo\ohm} &      \SI{3.29}{\kilo\ohm} \\ \hline
		C\sub{1}          & \SI{390}{\pico\farad} &   Non misurabile  \\ \hline
		
	\end{tabular}
	\caption{Valori nominali e misurati dei componenti passivi del circuito.}
	\label{tab:valori_componenti}
\end{table}

\noindent
Attraverso un generatore di forme d'onda sono stati applicati in ingresso al circuito dei segnali sinusoidali con ampiezza picco-picco pari a \SI{500}{\milli\volt} e frequenza compresa tra i \SI{100}{\hertz} e i \SI{10}{\mega\hertz} (\Fig \ref{fig:misure_oscilloscopio_1}).

\begin{figure}[h!]
	\centering
	a)
	
	\includegraphics[width=0.8\linewidth]{./ImageFiles/Laboratorio 1/TEK00001}	
\end{figure}
\begin{figure}[h!]
	\centering
	b)
	
	\includegraphics[width=0.8\linewidth]{./ImageFiles/Laboratorio 1/TEK00002}
\end{figure}
\begin{figure}[h!]
	\centering
	c)
	
	\includegraphics[width=0.8\linewidth]{./ImageFiles/Laboratorio 1/TEK00004}
	\caption{Misure del segnale in ingresso (linea gialla) e segnale in uscita (linea azzurra) al circuito. Si è utilizzato in ingresso un segnale sinusoidale con ampiezza picco-picco di \SI{500}{\milli\volt} e frequenza di \SI{1}{\hertz} (figura a), \SI{1}{\kilo\hertz} (figura b) e \SI{10}{\kilo\hertz} (figura c).}
	\label{fig:misure_oscilloscopio_1}
\end{figure}

\newpage
\noindent
Tramite le funzioni fornite dall'oscilloscopio, sono state ottenute le misure di ampiezza picco-picco del segnale in uscita al circuito e la differenza di fase tra il segnale in ingresso e quello in uscita. Grazie a questi valori sono stati ricostruiti i diagrammi di Bode di modulo e fase del circuito, riportati in figura \ref{fig:diagrammi_di_Bode}.
\begin{figure}[h!]
	\centering
	\includegraphics[width=0.80\linewidth]{./ImageFiles/Laboratorio 1/Diagrammi di Bode.png}
	\caption{Diagrammi di Bode del filtro passa basso attivo del primo ordine.}
	\label{fig:diagrammi_di_Bode}
\end{figure}
Dai diagrammi di Bode della funzione di trasferimento si può osservare il comportamento del filtro passa basso:
\begin{itemize}
	\item osservando il diagramma di Bode del modulo si può notare che fino alla frequenza di taglio, che è pari a \SI{10711}{\hertz}, il segnale in ingresso viene amplificato di circa \SI{20}{\decibel} (guadagno del circuito pari a 11.58, ossia \SI{21.27}{\decibel}), mentre per valori superiori alla frequenza di taglio viene attenuato con una pendenza pari a \SI{20}{\decibel}/dec;
	\item osservando il diagramma di Bode della fase si può notare che, in corrispondenza della frequenza di taglio, la fase è aumentata di circa 45$^{\circ}$rispetto alle frequenza più basse, mentre una decade dopo la frequenza di taglio raggiunge i -90$^{\circ}$.
\end{itemize}

\noindent
Tuttavia, per frequenze superiori ad \SI{1}{\mega\hertz}, il comportamento del circuito non è più quello ideale descritto dalla formula \ref{eq:1.3}: infatti, a tali frequenze subentrano dei limiti e dei comportamenti in frequenza particolari causati dalla banda limitata di funzionamento dell'amplificatore operazionale.
%\todo{nel datasheet \url{https://www.ti.com/lit/ds/symlink/tl071.pdf?ts=1665331321631&ref_url=https%253A%252F%252Fwww.ti.com%252Fproduct%252FTL071} è presente un grafico con guadagno in closed loop}



\todo{inserire prove di saturazione}
\todo{inserire prove di offset}
