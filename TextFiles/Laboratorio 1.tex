\chapter{Laboratorio 1}
In questa esperienza di laboratorio si è realizzato e analizzato il seguente circuito:
\begin{figure}[h!]
	\centering
	%\includegraphics[width=0.4\linewidth]{./OtherFiles/Laboratorio 1/xxxxx}
	\caption{Inserire schema e foto circuito}
	\label{fig:circuito}
\end{figure}

Il circuito realizza un filtro passa basso attivo, utilizzando un amplificatore operazionale retroazionato negativamente. Infatti, è possibile ricavare la funzione di trasferimento del circuito tramite un bilancio delle correnti al nodo V\super{-}\todo{Magari inserire nello schema un label al nodo V- e anche il nodo vout}. Si consideri v\sub{in} come un generatore ideale di tensione applicato in ingresso al circuito. Indicando con Z\sub{1} l'impedenza equivalente del parallelo tra R\sub{1} e C\sub{1} e con Z\sub{2} l'impedenza della resistenza R\sub{2}, si può ottenere la funzione di trasferimento del circuito:
\begin{equation}
	v_{out}=-\frac{Z_1}{Z_2}v_{in}=-\frac{1}{R_2}\frac{R_1}{1+j w R_1 C_1} vin,
\end{equation}
da cui si ricava
\begin{equation}
	T=\frac{v_{out}}{v_{in}}=-\frac{R_1}{R_2}\frac{1}{1+j w R_1 C_1}.
\end{equation}
Questa funzione di trasferimento corrisponde a un filtro passa basso. Infatti, è possibile calcolare l'andamento del modulo e della fase in funzione di $\omega$:
\begin{equation}
	\begin{split}
		|T|&=\frac{R_1}{R_2}\frac{1}{\sqrt{1+(wR_1C_1)^2}} \\
		\angle T&=\SI{180}{\degree}-arctan(\omega R_1 C_1).
	\end{split}
\end{equation}
Per comprendere il comportamento del modulo e della fase in funzione della frequenza del segnale applicato in ingresso, è necessario analizzare i termini dipendenti da $\omega$ nelle due equazioni. Per quanto riguarda l'espressione del modulo della funzione di trasferimento, il termine $\sqrt{1+(wR_1C_1)^2}.... $ \todo{inserire le due frecce per i limiti} mentre nell'espressione della fase, il termine $arctan(\omega R_1 C_1)....$ \todo{ut supra}. Il modulo quindi tende a $\frac{R_1}{R_2}$ per segnali in ingresso a bassa frequenza, mentre tende a zero per segnali in ingresso ad alta frequenza, mentre la fase è pari a circa \SI{180}{\degree} per segnali a bassa frequenza, e tende a \SI{90}{\degree} per segnali ad alta frequenza.\todo{da sistemare questa parte}
Si determina quindi un comportamento di un filtro passa basso invertente del primo ordine, con frequenza di taglio pari a $f=\frac{1}{2\pi R_1C_1}$.

I valori dei componenti passivi sono stati scelti per soddisfare i requisiti di prestazioni del filtro. In particolare, veniva richiesto un guadagno di un fattore 10 con frequenza di taglio pari a \SI{10}{\kilo\hertz}. I valori scelti ... (\todo{inserire tabella valori con valori nominali/misurati dei componenti}).

\todo{Inserire grafici matlab}
\todo{inserire prove di saturazione}
\todo{inserire prove di offset}